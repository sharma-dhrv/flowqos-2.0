\section{Introduction}
\label{sec:introduction}

%why we need FlowQos?
In modern times, user devices connected to broadband access networks, run an assortment of applications that exchange network traffic with remote servers and other devices over the Internet. Since the upstream and downstream throughput is generally limited, these traffics will compete for relatively scarce bandwidth resources. 

However, traffic from one application might not share the same characteristics as the traffic created from another application. To a large extent, it is the end-user's requirements that play a significant role in deciding the nature of such applications and hence the kind of network traffic they send and receive. For instance, a user's expectation from a VoIP call, video streaming and gaming applications is that their experience remains seamless and of high quality during their use of the application, requiring the traffic to be sent out at near constant rates with low delay and high reliability paths over the network. Whereas, in certain use cases such as data backup to the cloud and system updates, it is expected that the operation must completed eventually - even when the user is not actively using the application. Traffic from such applications, in contrast to VoIP and video streaming, does not come with any hard deadlines or network requirements.

While the nature of traffic varies so widely, the network devices today are, to a large extent, agnostic to such subtleties in the nature of network traffic and tend to handle it in the same way. Even though later kinds of applications discussed above do not face any issues that directly affect the user, the former kind might be impacted severely due to the effect on network dynamics leaving the user with a sour experience. 

%previous work
Research over the past years has resulted in the use of various metrics such as packet delays, jitter, available bandwidth, frame rate etc. to quantify user’s experience in some way. One possible way to deal with limited throughput is to configure the network routers to prioritize some specific applications' traffic flows (e.g. video, VoIP etc) over others (e.g. data backup, file upload etc). It will effectively improve Quality of Service (QoS). However, for some reason, previous work on QoS mechanisms have not been deployed in broadband access networks~\cite{Seddiki2014}. The emergence of software-defined networking (SDN) gives more possibilities to solve this problem. One approach to deploying QoS in broadband access networks is to utilize the advantages of SDN that separates the network's control logic and forwarding planes. We can migrate the functions that perform QoS both application identification and router-level configuration to separate control logic, and design a front-end client (e.g., webpage) at high levels of abstraction to let users to assign bandwidth to each identified application according to their own preference. Once users set up their preference, the front-end will install the QoS configuration into the home routers~\cite{Seddiki2014}. This approach makes it easy for user to configure priorities and facilitates more sophisticated per-flow application based QoS. 

%Our contribution
\subsection{Our Contribution}
First, we deploy the FlowQoS implementation of pair of virtual switches for improving QoS using classification and traffic policing, within a virtual machine (emulating our end-host) with Internet connection instead of the discussed hardware implementation in the original FLowQoS paper. Second, we validate the results presented by FlowQoS~\cite{Seddiki2014} under similar scenarios that they used but eveluating the effects using different metrics and applications. Finally, we implement and evaluate an architecture that utilizes Linux’s advanced routing and traffic control to implement the idea of FlowQoS while overcoming the limitation of under-utilization of available bandwidth. 

%The rest of the paper
\subsection{Outline}
The rest of the paper is organized in the following way: We discussed related work in QoS, as well as SDN-based solutions for home and broadband access networks in ~\xref{sec:related-works}. In ~\xref{sec:background} we discuss the difference between rate limiting using traffic policing and traffic shaping. The motivation of FlowQoS is presented in ~\xref{sec:motivation}. ~\xref{sec:design} describes various components of the FlowQoS implementation. In ~\xref{sec:evaluation}, we evaluate FlowQoS for video streaming and VoIP applications in the context of competing TCP flows. Finally, we discuss our results, future work and other open research avenues in ~\xref{sec:discussion} and conclude in \xref{sec:conclusion}.